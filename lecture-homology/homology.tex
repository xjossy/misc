\documentclass[8pt]{beamer}

\usetheme{Madrid}

\usepackage[utf8]{inputenc}
\usepackage[english,russian]{babel}
\usepackage{amsmath}
\usepackage{mathtools}
\usepackage{xcolor}

\DeclareMathOperator{\Ker}{Ker}
\DeclareMathOperator{\Ima}{Im}

\title{Гомологии. Обзор}
\subtitle{Введение в цепные, симплициальные, сингулярные и клеточные гомологии}


%\usetheme{lucid}
\begin{document}
	\frame {
		\titlepage
	}
	\frame {
		\frametitle{Начальные сведения из алгебры}
		\large
		Отображение групп (объектов абелевой категории) $f: A\to B$ имеет ядро $\Ker f\subset A$ и образ $\Ima f\subset B$.
		\begin{itemize}
		\item Для групп $\Ker f = \{x\in A| f(x)=0\}$, $\Ima f = f(A)$.
		\item \pause Условие $\Ker f=0$ означает инъективность $f$ ($f(x)\ne f(y)$ при $x\ne y$)
		\item Условие $\Ima f=B$ означает сюрьективность $f$ ($\forall b\in B\exists a\in A f(a)=b$)
		\item \pause Что означает условие $\Ker f=A$, $\Ima f=0$?
		\end{itemize}

		\kern1cm
		\pause
		Важную роль будут играть свободные абелевы группы.
		\pause
		\begin{itemize}
		\item Свободная (абелева) группа с образующими из множества $A$ (обозн. $\mathbb Z^A$) это множество отображений $A\to \mathbb Z$, имеющих конечное количество элементов с ненулевым образом. 
		\item Для каждого $a\in A$ в $\mathbb Z^A$ есть образующая $f: A\to\mathbb Z, f(x) = \delta(x,a)$
		\item Образующую, соответствующую элементу $a$,  будем обозначать $a$
		\item Гомоморфизм $\mathbb Z^A\to G$ в любую группу достаточно (произвольно) задать на образующих
		\end{itemize}
	}
	\frame {
		\large
		\frametitle{Точные последовательности}
		\begin{columns}
		\column{0.45\textwidth}

		\begin{block}{Определение}
		\[\cdots \xrightarrow{f_{n-2}} U_{n-1} \xrightarrow{f_{n-1}} U_{n} \xrightarrow{f_{n}} U_{n+1}\to\cdots\]
		Ограниченная или неограниченная последовательность абелевых групп $F_n$ и их гомоморфизмов $f_n: F_n\to F_{n+1}$ (в общем случае, объектов абелевой категории) называется точной в члене $n$, если
		\[\Ima f_{n-1} = \Ker f_{n}.\]
		И просто точной, если она точна во всех членах.
		\end{block}

		\column{0.45\textwidth}

		\pause

		Что означает точность в этих примерах?

		\begin{itemize}
		\item $0 \to A \to C$
		\pause\item $C \to B \to 0$
		\pause\item $0 \to F \to C \to B$
		\pause\item $G \to C \to B \to 0$
		\pause\item ({\it Короткая точная последовательность}) $0 \to A \to C \to B \to 0$
		\pause\item ({\it Левая резольвента})\\$\cdots \to P_2 \to P_1 \to F \to 0$
		\end{itemize}


		\end{columns}
	}
	\frame {
\large
		\frametitle{Цепные комплексы}
		А что, если ослабить условие $\Ima f_{n} = \Ker f_{n+1}$ до $\Ima f_{n} \subset \Ker f_{n+1}$? Такая последовательность будет называться {\it коцепной комплекс}.

		Если перенумеровать группы в обратном порядке, то такой комплекс называют {\it цепным}, гомоморфизмы традиционно обозначают $\partial_n:F_n\to F_{n-1}$.

		\begin{block}{Лемма}
		Условие $\Ima\partial_{n+1} \subset \Ker\partial_{n}$ эквивалентно $\partial_{n}\partial_{n+1} = 0.$
		\end{block}

		\pause

		Поскольку последовательность уже не точная, можно измерить, насколько сильно она не точная. Для этого положим $H_n = \Ker\partial_{n} / \Ima\partial_{n+1}$

		\pause

		\begin{block}{Итог}
		Цепной комплекс $$\cdots \xleftarrow{\partial_{n-1}} K_{n-1} \xleftarrow{\partial_{n}} K_{n} \xleftarrow{\partial_{n+1}} K_{n+1}\leftarrow\cdots$$

		\pause Элементы $K_n$~-- \textit{цепи}, размерности $n$. 
		
		\pause Элементы $B_n = \Ima\partial_{n+1}\subset K_n$~-- \textit{границы}. 

		Элементы $Z_n = \Ker\partial_{n}\subset K_n$~-- \textit{циклы}.
		
		\pause Условие комплекса: $B_n\subset Z_n$. \textit{Гомологии} $H_n=Z_n/B_n$.
		\end{block}

		\pause Чему равны гомологии точной последовательности?
	}

	\frame {
		\frametitle{Схема гомологий}
		
		\begin{figure}
		    \centering
		    \def\svgwidth{1\columnwidth}
		    \input{hom-scheme.pdf_tex}
		\end{figure}
	}

	\frame {
		\frametitle{Геометрические примеры комплексов}
		
		\Large{Все дальнейшие примеры гомологий описывают только построение цепного комплекса. Сами гомологии вычисляются в соответствии с определением гомологий цепного комплекса}

		\begin{itemize}
		\item На $X$ вводится некая геометрическая структура
		\item \pause По этой стуктуре строится цепной комплекс
		\item \pause У этого комплекса вычисляются гомологии и объявляются гомологиями $X$
		\item Обычно, геометрические комплексы состоят из свободных (абелевых) групп $\mathbb Z^A$. 
		%\item Гомоморфизм свободной группы достаточно (произвольно) задать на образующих
		\item Вместо $\mathbb Z$ можно использовать поле, например, $\mathbb R$ или даже $\mathbb F_2$, в этом случае, вычисление упрощается, однако, теряется часть информации.
		\end{itemize}

		\pause
		\begin{block}{Проблема}
		А если ввести геометрическую структуру иным образом, получатся ли гомологии такими же?

		\pause Это называется инвариантностью гомологий.
		\end{block}

	}

	\frame {
		\frametitle{Симплициальные гомологии}
		\begin{columns}
		\column{0.45\textwidth}
		
		\begin{itemize}
		\item У топологического пространства с заданной триангуляцией
		\item Требуют задания явной структуры
		\item Имеют формальное обобщение
		\item Можно вычислять гомологии, хотя и трудоёмко
		\item Просто доказать условие комплекса $\partial\partial=0$
		\item Инвариантность доказать сложно
		\end{itemize}

		\pause
		Симплекс $\Delta_n$ размерности $n$ - выпуклая оболочка $n+1$ точки общего положения
		\begin{figure}
		    \centering
		    \def\svgwidth{.6\columnwidth}
		    %LaTeX with PSTricks extensions
%%Creator: 0.91_64bit
%%Please note this file requires PSTricks extensions
\psset{xunit=.5pt,yunit=.5pt,runit=.5pt}
\begin{pspicture}(300.57907104,118.22928631)
{
\newrgbcolor{curcolor}{0 0 0}
\pscustom[linewidth=1,linecolor=curcolor]
{
\newpath
\moveto(124.710632,96.91634631)
\lineto(170.167502,18.18274731)
\lineto(79.253772,18.18274731)
\lineto(124.710632,96.91634631)
}
}
{
\newrgbcolor{curcolor}{0 0 0}
\pscustom[linestyle=none,fillstyle=solid,fillcolor=curcolor]
{
\newpath
\moveto(84.77067017,7.47685065)
\curveto(84.77067017,5.51884284)(84.46305298,4.08085456)(83.84781861,3.16288581)
\curveto(83.23746704,2.24979987)(82.28776001,1.7932569)(80.99869751,1.7932569)
\curveto(79.69010376,1.7932569)(78.73307251,2.25712409)(78.12760376,3.18485846)
\curveto(77.52701783,4.11259284)(77.22672486,5.53837409)(77.22672486,7.46220221)
\curveto(77.22672486,9.40067877)(77.53190064,10.83134284)(78.1422522,11.7541944)
\curveto(78.75260376,12.68192877)(79.7047522,13.14579596)(80.99869751,13.14579596)
\curveto(82.30729126,13.14579596)(83.26188111,12.67460456)(83.86246704,11.73222174)
\curveto(84.46793579,10.79472174)(84.77067017,9.37626471)(84.77067017,7.47685065)
\closepath
\moveto(82.84440064,4.15165534)
\curveto(83.01529908,4.54716315)(83.13004517,5.01103034)(83.18863892,5.5432569)
\curveto(83.25211548,6.08036627)(83.28385376,6.72489752)(83.28385376,7.47685065)
\curveto(83.28385376,8.21903815)(83.25211548,8.8635694)(83.18863892,9.4104444)
\curveto(83.13004517,9.9573194)(83.01285767,10.42118659)(82.83707642,10.80204596)
\curveto(82.66617798,11.17802252)(82.43180298,11.46122565)(82.13395142,11.65165534)
\curveto(81.84098267,11.84208502)(81.4625647,11.93729987)(80.99869751,11.93729987)
\curveto(80.53971314,11.93729987)(80.15885376,11.84208502)(79.85611939,11.65165534)
\curveto(79.55826783,11.46122565)(79.32145142,11.17313971)(79.14567017,10.78739752)
\curveto(78.97965454,10.4260694)(78.86490845,9.95487799)(78.80143189,9.37382331)
\curveto(78.74283814,8.79276862)(78.71354126,8.15556159)(78.71354126,7.46220221)
\curveto(78.71354126,6.70048346)(78.74039673,6.06327643)(78.79410767,5.55058112)
\curveto(78.84781861,5.03788581)(78.9625647,4.57890143)(79.13834595,4.17362799)
\curveto(79.29947876,3.79276862)(79.52652954,3.50224127)(79.81949829,3.30204596)
\curveto(80.11734986,3.10185065)(80.51041626,3.00175299)(80.99869751,3.00175299)
\curveto(81.45768189,3.00175299)(81.83854126,3.09696784)(82.14127564,3.28739752)
\curveto(82.44401001,3.47782721)(82.67838501,3.76591315)(82.84440064,4.15165534)
\closepath
}
}
{
\newrgbcolor{curcolor}{0 0 0}
\pscustom[linestyle=none,fillstyle=solid,fillcolor=curcolor]
{
\newpath
\moveto(176.08664917,2.02030768)
\lineto(170.18332886,2.02030768)
\lineto(170.18332886,3.13358893)
\lineto(172.45383667,3.13358893)
\lineto(172.45383667,10.44315924)
\lineto(170.18332886,10.44315924)
\lineto(170.18332886,11.43925299)
\curveto(170.49094605,11.43925299)(170.82053589,11.46366706)(171.17209839,11.51249518)
\curveto(171.52366089,11.56620612)(171.78977417,11.64188971)(171.97043824,11.73954596)
\curveto(172.19504761,11.86161627)(172.37082886,12.01542487)(172.49778199,12.20097174)
\curveto(172.62961792,12.39140143)(172.70530152,12.64530768)(172.72483277,12.96269049)
\lineto(173.86008667,12.96269049)
\lineto(173.86008667,3.13358893)
\lineto(176.08664917,3.13358893)
\lineto(176.08664917,2.02030768)
\closepath
}
}
{
\newrgbcolor{curcolor}{0 0 0}
\pscustom[linestyle=none,fillstyle=solid,fillcolor=curcolor]
{
\newpath
\moveto(127.21172547,103.03555884)
\lineto(119.82891297,103.03555884)
\lineto(119.82891297,104.56632056)
\lineto(121.3669989,105.88467993)
\curveto(121.88457703,106.32413306)(122.36553406,106.76114478)(122.80987,107.19571509)
\curveto(123.74737,108.10391821)(124.38945984,108.82413306)(124.73613953,109.35635962)
\curveto(125.08281922,109.893469)(125.25615906,110.47208228)(125.25615906,111.09219946)
\curveto(125.25615906,111.65860571)(125.06817078,112.10050025)(124.69219422,112.41788306)
\curveto(124.32110047,112.74014868)(123.80108094,112.9012815)(123.13213562,112.9012815)
\curveto(122.68779969,112.9012815)(122.20684265,112.8231565)(121.68926453,112.6669065)
\curveto(121.1716864,112.5106565)(120.66631531,112.27139868)(120.17315125,111.94913306)
\lineto(120.09990906,111.94913306)
\lineto(120.09990906,113.487219)
\curveto(120.44658875,113.65811743)(120.90801453,113.81436743)(121.4841864,113.955969)
\curveto(122.06524109,114.09757056)(122.62676453,114.16837134)(123.16875672,114.16837134)
\curveto(124.28692078,114.16837134)(125.16338562,113.89737525)(125.79815125,113.35538306)
\curveto(126.43291687,112.81827368)(126.75029969,112.08829321)(126.75029969,111.16544165)
\curveto(126.75029969,110.75040259)(126.69658875,110.362219)(126.58916687,110.00089087)
\curveto(126.48662781,109.64444556)(126.33281922,109.30509009)(126.12774109,108.98282446)
\curveto(125.9373114,108.68009009)(125.71270203,108.38223853)(125.45391297,108.08926978)
\curveto(125.20000672,107.79630103)(124.88994812,107.471594)(124.52373719,107.11514868)
\curveto(124.00127625,106.60245337)(123.46172547,106.1044065)(122.90508484,105.62100806)
\curveto(122.34844422,105.14249243)(121.82842469,104.6981565)(121.34502625,104.28800025)
\lineto(127.21172547,104.28800025)
\lineto(127.21172547,103.03555884)
\closepath
}
}
{
\newrgbcolor{curcolor}{0 0 0}
\pscustom[linewidth=0.96624851,linecolor=curcolor,linestyle=dashed,dash=0.96624851 1.93249702]
{
\newpath
\moveto(268.395962,43.51468031)
\lineto(298.198242,18.32572731)
\lineto(202.228222,18.32572731)
\lineto(268.395962,43.51468031)
}
}
{
\newrgbcolor{curcolor}{0 0 0}
\pscustom[linewidth=1,linecolor=curcolor]
{
\newpath
\moveto(259.304592,102.02540831)
\lineto(202.228222,18.32572731)
}
}
{
\newrgbcolor{curcolor}{0 0 0}
\pscustom[linewidth=1,linecolor=curcolor,linestyle=dashed,dash=1 2]
{
\newpath
\moveto(259.304592,102.02540831)
\lineto(298.198242,18.32572731)
}
}
{
\newrgbcolor{curcolor}{0 0 0}
\pscustom[linewidth=1,linecolor=curcolor,linestyle=dashed,dash=1 2]
{
\newpath
\moveto(259.304592,102.02540831)
\lineto(268.395962,43.51468031)
}
}
{
\newrgbcolor{curcolor}{0 0 0}
\pscustom[linewidth=1,linecolor=curcolor]
{
\newpath
\moveto(259.304592,102.02540831)
\lineto(298.198242,18.32572731)
}
}
{
\newrgbcolor{curcolor}{0 0 0}
\pscustom[linewidth=1,linecolor=curcolor]
{
\newpath
\moveto(298.198242,18.32572731)
\lineto(298.198242,18.32572731)
\lineto(298.198242,18.32572731)
\lineto(298.198242,18.32572731)
\lineto(202.228222,18.32572731)
}
}
{
\newrgbcolor{curcolor}{0 0 0}
\pscustom[linestyle=none,fillstyle=solid,fillcolor=curcolor]
{
\newpath
\moveto(202.19202637,8.48699774)
\curveto(202.19202637,6.52898993)(201.88440918,5.09100165)(201.26917481,4.1730329)
\curveto(200.65882325,3.25994696)(199.70911621,2.80340399)(198.42005371,2.80340399)
\curveto(197.11145996,2.80340399)(196.15442871,3.26727118)(195.54895996,4.19500556)
\curveto(194.94837403,5.12273993)(194.64808106,6.54852118)(194.64808106,8.47234931)
\curveto(194.64808106,10.41082587)(194.95325684,11.84148993)(195.5636084,12.76434149)
\curveto(196.17395996,13.69207587)(197.1261084,14.15594306)(198.42005371,14.15594306)
\curveto(199.72864746,14.15594306)(200.68323731,13.68475165)(201.28382325,12.74236884)
\curveto(201.889292,11.80486884)(202.19202637,10.38641181)(202.19202637,8.48699774)
\closepath
\moveto(200.26575684,5.16180243)
\curveto(200.43665528,5.55731024)(200.55140137,6.02117743)(200.60999512,6.55340399)
\curveto(200.67347168,7.09051337)(200.70520996,7.73504462)(200.70520996,8.48699774)
\curveto(200.70520996,9.22918524)(200.67347168,9.87371649)(200.60999512,10.42059149)
\curveto(200.55140137,10.96746649)(200.43421387,11.43133368)(200.25843262,11.81219306)
\curveto(200.08753418,12.18816962)(199.85315918,12.47137274)(199.55530762,12.66180243)
\curveto(199.26233887,12.85223212)(198.8839209,12.94744696)(198.42005371,12.94744696)
\curveto(197.96106934,12.94744696)(197.58020996,12.85223212)(197.27747559,12.66180243)
\curveto(196.97962403,12.47137274)(196.74280762,12.18328681)(196.56702637,11.79754462)
\curveto(196.40101075,11.43621649)(196.28626465,10.96502509)(196.22278809,10.3839704)
\curveto(196.16419434,9.80291571)(196.13489746,9.16570868)(196.13489746,8.47234931)
\curveto(196.13489746,7.71063056)(196.16175293,7.07342352)(196.21546387,6.56072821)
\curveto(196.26917481,6.0480329)(196.3839209,5.58904852)(196.55970215,5.18377509)
\curveto(196.72083496,4.80291571)(196.94788575,4.51238837)(197.2408545,4.31219306)
\curveto(197.53870606,4.11199774)(197.93177246,4.01190009)(198.42005371,4.01190009)
\curveto(198.87903809,4.01190009)(199.25989746,4.10711493)(199.56263184,4.29754462)
\curveto(199.86536621,4.48797431)(200.09974121,4.77606024)(200.26575684,5.16180243)
\closepath
}
}
{
\newrgbcolor{curcolor}{0 0 0}
\pscustom[linestyle=none,fillstyle=solid,fillcolor=curcolor]
{
\newpath
\moveto(300.57906555,-0.00000177)
\lineto(294.67574524,-0.00000177)
\lineto(294.67574524,1.11327948)
\lineto(296.94625305,1.11327948)
\lineto(296.94625305,8.42284979)
\lineto(294.67574524,8.42284979)
\lineto(294.67574524,9.41894354)
\curveto(294.98336243,9.41894354)(295.31295227,9.44335761)(295.66451477,9.49218573)
\curveto(296.01607727,9.54589667)(296.28219055,9.62158026)(296.46285462,9.71923651)
\curveto(296.68746399,9.84130683)(296.86324524,9.99511542)(296.99019837,10.18066229)
\curveto(297.1220343,10.37109198)(297.1977179,10.62499823)(297.21724915,10.94238104)
\lineto(298.35250305,10.94238104)
\lineto(298.35250305,1.11327948)
\lineto(300.57906555,1.11327948)
\lineto(300.57906555,-0.00000177)
\closepath
}
}
{
\newrgbcolor{curcolor}{0 0 0}
\pscustom[linestyle=none,fillstyle=solid,fillcolor=curcolor]
{
\newpath
\moveto(260.79552674,47.47717108)
\lineto(253.41271424,47.47717108)
\lineto(253.41271424,49.0079328)
\lineto(254.95080017,50.32629218)
\curveto(255.4683783,50.7657453)(255.94933533,51.20275702)(256.39367127,51.63732733)
\curveto(257.33117127,52.54553046)(257.97326111,53.2657453)(258.3199408,53.79797186)
\curveto(258.66662049,54.33508124)(258.83996033,54.91369452)(258.83996033,55.53381171)
\curveto(258.83996033,56.10021796)(258.65197205,56.54211249)(258.27599549,56.8594953)
\curveto(257.90490174,57.18176093)(257.38488221,57.34289374)(256.71593689,57.34289374)
\curveto(256.27160096,57.34289374)(255.79064392,57.26476874)(255.2730658,57.10851874)
\curveto(254.75548767,56.95226874)(254.25011658,56.71301093)(253.75695252,56.3907453)
\lineto(253.68371033,56.3907453)
\lineto(253.68371033,57.92883124)
\curveto(254.03039002,58.09972968)(254.4918158,58.25597968)(255.06798767,58.39758124)
\curveto(255.64904236,58.5391828)(256.2105658,58.60998358)(256.75255799,58.60998358)
\curveto(257.87072205,58.60998358)(258.74718689,58.33898749)(259.38195252,57.7969953)
\curveto(260.01671814,57.25988593)(260.33410096,56.52990546)(260.33410096,55.6070539)
\curveto(260.33410096,55.19201483)(260.28039002,54.80383124)(260.17296814,54.44250311)
\curveto(260.07042908,54.0860578)(259.91662049,53.74670233)(259.71154236,53.42443671)
\curveto(259.52111267,53.12170233)(259.2965033,52.82385077)(259.03771424,52.53088202)
\curveto(258.78380799,52.23791327)(258.47374939,51.91320624)(258.10753846,51.55676093)
\curveto(257.58507752,51.04406561)(257.04552674,50.54601874)(256.48888611,50.0626203)
\curveto(255.93224549,49.58410468)(255.41222596,49.13976874)(254.92882752,48.72961249)
\lineto(260.79552674,48.72961249)
\lineto(260.79552674,47.47717108)
\closepath
}
}
{
\newrgbcolor{curcolor}{0 0 0}
\pscustom[linestyle=none,fillstyle=solid,fillcolor=curcolor]
{
\newpath
\moveto(260.82592224,112.34793868)
\curveto(261.06029724,112.13797774)(261.25316834,111.87430586)(261.40453553,111.55692305)
\curveto(261.55590271,111.23954024)(261.63158631,110.82938399)(261.63158631,110.3264543)
\curveto(261.63158631,109.82840743)(261.54125428,109.37186446)(261.36059021,108.9568254)
\curveto(261.17992615,108.54178633)(260.9260199,108.18045821)(260.59887146,107.87284102)
\curveto(260.23266053,107.53104415)(259.80053162,107.2771379)(259.30248474,107.11112227)
\curveto(258.80932068,106.94998946)(258.26732849,106.86942305)(257.67650818,106.86942305)
\curveto(257.07103943,106.86942305)(256.47533631,106.94266524)(255.88939881,107.08914961)
\curveto(255.30346131,107.23075118)(254.82250428,107.38700118)(254.44652771,107.55789961)
\lineto(254.44652771,109.08866133)
\lineto(254.55639099,109.08866133)
\curveto(254.97143006,108.81522383)(255.45971131,108.58817305)(256.02123474,108.40750899)
\curveto(256.58275818,108.22684493)(257.12475037,108.1365129)(257.64721131,108.1365129)
\curveto(257.95482849,108.1365129)(258.28197693,108.18778243)(258.62865662,108.29032149)
\curveto(258.97533631,108.39286055)(259.25609803,108.54422774)(259.47094178,108.74442305)
\curveto(259.69555115,108.9592668)(259.86156678,109.19608321)(259.96898865,109.45487227)
\curveto(260.08129334,109.71366133)(260.13744568,110.04080977)(260.13744568,110.43631758)
\curveto(260.13744568,110.82694258)(260.07396912,111.14920821)(259.94701599,111.40311446)
\curveto(259.82494568,111.66190352)(259.65404724,111.86454024)(259.43432068,112.01102461)
\curveto(259.21459412,112.1623918)(258.94848084,112.26493086)(258.63598084,112.3186418)
\curveto(258.32348084,112.37723555)(257.98656678,112.40653243)(257.62523865,112.40653243)
\lineto(256.96605896,112.40653243)
\lineto(256.96605896,113.62235274)
\lineto(257.47875428,113.62235274)
\curveto(258.22094178,113.62235274)(258.81176209,113.77616133)(259.25121521,114.08377852)
\curveto(259.69555115,114.39627852)(259.91771912,114.85038008)(259.91771912,115.44608321)
\curveto(259.91771912,115.70975508)(259.86156678,115.93924727)(259.74926209,116.13455977)
\curveto(259.6369574,116.33475508)(259.4807074,116.4983293)(259.28051209,116.62528243)
\curveto(259.07055115,116.75223555)(258.84594178,116.84012618)(258.60668396,116.8889543)
\curveto(258.36742615,116.93778243)(258.09643006,116.96219649)(257.79369568,116.96219649)
\curveto(257.32982849,116.96219649)(256.83666443,116.87918868)(256.31420349,116.71317305)
\curveto(255.79174256,116.54715743)(255.29857849,116.31278243)(254.83471131,116.01004805)
\lineto(254.76146912,116.01004805)
\lineto(254.76146912,117.54080977)
\curveto(255.10814881,117.71170821)(255.56957459,117.86795821)(256.14574646,118.00955977)
\curveto(256.72680115,118.15604415)(257.28832459,118.22928633)(257.83031678,118.22928633)
\curveto(258.36254334,118.22928633)(258.83129334,118.18045821)(259.23656678,118.08280196)
\curveto(259.64184021,117.98514571)(260.00805115,117.82889571)(260.33519959,117.61405196)
\curveto(260.68676209,117.37967696)(260.95287537,117.09647383)(261.13353943,116.76444258)
\curveto(261.31420349,116.43241133)(261.40453553,116.04422774)(261.40453553,115.5998918)
\curveto(261.40453553,114.99442305)(261.18969178,114.4646379)(260.76000428,114.01053633)
\curveto(260.33519959,113.56131758)(259.8322699,113.27811446)(259.25121521,113.16092696)
\lineto(259.25121521,113.0583879)
\curveto(259.48559021,113.0193254)(259.7541449,112.93631758)(260.05687928,112.80936446)
\curveto(260.35961365,112.68729415)(260.61596131,112.53348555)(260.82592224,112.34793868)
\closepath
}
}
{
\newrgbcolor{curcolor}{1 0 0}
\pscustom[linestyle=none,fillstyle=solid,fillcolor=curcolor]
{
\newpath
\moveto(81.57519197,18.18274512)
\curveto(81.57519197,16.90065554)(80.53585301,15.86131658)(79.25376343,15.86131658)
\curveto(77.97167385,15.86131658)(76.93233489,16.90065554)(76.93233489,18.18274512)
\curveto(76.93233489,19.4648347)(77.97167385,20.50417366)(79.25376343,20.50417366)
\curveto(80.53585301,20.50417366)(81.57519197,19.4648347)(81.57519197,18.18274512)
\closepath
}
}
{
\newrgbcolor{curcolor}{0 0 0.50196081}
\pscustom[linestyle=none,fillstyle=solid,fillcolor=curcolor]
{
\newpath
\moveto(172.48891877,18.18274512)
\curveto(172.48891877,16.90065554)(171.44957982,15.86131658)(170.16749024,15.86131658)
\curveto(168.88540066,15.86131658)(167.8460617,16.90065554)(167.8460617,18.18274512)
\curveto(167.8460617,19.4648347)(168.88540066,20.50417366)(170.16749024,20.50417366)
\curveto(171.44957982,20.50417366)(172.48891877,19.4648347)(172.48891877,18.18274512)
\closepath
}
}
{
\newrgbcolor{curcolor}{0.66666669 0.53333336 0}
\pscustom[linestyle=none,fillstyle=solid,fillcolor=curcolor]
{
\newpath
\moveto(127.03205537,96.91634574)
\curveto(127.03205537,95.63425616)(125.99271641,94.5949172)(124.71062683,94.5949172)
\curveto(123.42853726,94.5949172)(122.3891983,95.63425616)(122.3891983,96.91634574)
\curveto(122.3891983,98.19843531)(123.42853726,99.23777427)(124.71062683,99.23777427)
\curveto(125.99271641,99.23777427)(127.03205537,98.19843531)(127.03205537,96.91634574)
\closepath
}
}
{
\newrgbcolor{curcolor}{1 0 0}
\pscustom[linestyle=none,fillstyle=solid,fillcolor=curcolor]
{
\newpath
\moveto(204.54966401,18.32571997)
\curveto(204.54966401,17.04363039)(203.51032506,16.00429144)(202.22823548,16.00429144)
\curveto(200.9461459,16.00429144)(199.90680694,17.04363039)(199.90680694,18.32571997)
\curveto(199.90680694,19.60780955)(200.9461459,20.64714851)(202.22823548,20.64714851)
\curveto(203.51032506,20.64714851)(204.54966401,19.60780955)(204.54966401,18.32571997)
\closepath
}
}
{
\newrgbcolor{curcolor}{0 0 0.50196081}
\pscustom[linestyle=none,fillstyle=solid,fillcolor=curcolor]
{
\newpath
\moveto(300.51966523,18.32571997)
\curveto(300.51966523,17.04363039)(299.48032628,16.00429144)(298.1982367,16.00429144)
\curveto(296.91614712,16.00429144)(295.87680816,17.04363039)(295.87680816,18.32571997)
\curveto(295.87680816,19.60780955)(296.91614712,20.64714851)(298.1982367,20.64714851)
\curveto(299.48032628,20.64714851)(300.51966523,19.60780955)(300.51966523,18.32571997)
\closepath
}
}
{
\newrgbcolor{curcolor}{0.66666669 0.53333336 0}
\pscustom[linestyle=none,fillstyle=solid,fillcolor=curcolor]
{
\newpath
\moveto(261.62601899,102.02540602)
\curveto(261.62601899,100.74331644)(260.58668004,99.70397749)(259.30459046,99.70397749)
\curveto(258.02250088,99.70397749)(256.98316192,100.74331644)(256.98316192,102.02540602)
\curveto(256.98316192,103.3074956)(258.02250088,104.34683456)(259.30459046,104.34683456)
\curveto(260.58668004,104.34683456)(261.62601899,103.3074956)(261.62601899,102.02540602)
\closepath
}
}
{
\newrgbcolor{curcolor}{0 0.50196081 0}
\pscustom[linestyle=none,fillstyle=solid,fillcolor=curcolor]
{
\newpath
\moveto(270.71738862,43.51467719)
\curveto(270.71738862,42.23258761)(269.67804966,41.19324865)(268.39596009,41.19324865)
\curveto(267.11387051,41.19324865)(266.07453155,42.23258761)(266.07453155,43.51467719)
\curveto(266.07453155,44.79676677)(267.11387051,45.83610572)(268.39596009,45.83610572)
\curveto(269.67804966,45.83610572)(270.71738862,44.79676677)(270.71738862,43.51467719)
\closepath
}
}
{
\newrgbcolor{curcolor}{0 0 0}
\pscustom[linewidth=1,linecolor=curcolor]
{
\newpath
\moveto(55.378022,56.38153931)
\lineto(2.321426,56.38153931)
}
}
{
\newrgbcolor{curcolor}{0 0 0}
\pscustom[linestyle=none,fillstyle=solid,fillcolor=curcolor]
{
\newpath
\moveto(7.83833527,45.67563452)
\curveto(7.83833527,43.71762671)(7.53071808,42.27963843)(6.91548371,41.36166968)
\curveto(6.30513214,40.44858374)(5.35542511,39.99204077)(4.06636261,39.99204077)
\curveto(2.75776886,39.99204077)(1.80073761,40.45590796)(1.19526886,41.38364234)
\curveto(0.59468293,42.31137671)(0.29438996,43.73715796)(0.29438996,45.66098609)
\curveto(0.29438996,47.59946265)(0.59956574,49.03012671)(1.2099173,49.95297827)
\curveto(1.82026886,50.88071265)(2.7724173,51.34457984)(4.06636261,51.34457984)
\curveto(5.37495636,51.34457984)(6.32954621,50.87338843)(6.93013214,49.93100562)
\curveto(7.53560089,48.99350562)(7.83833527,47.57504859)(7.83833527,45.67563452)
\closepath
\moveto(5.91206574,42.35043921)
\curveto(6.08296418,42.74594702)(6.19771027,43.20981421)(6.25630402,43.74204077)
\curveto(6.31978058,44.27915015)(6.35151886,44.9236814)(6.35151886,45.67563452)
\curveto(6.35151886,46.41782202)(6.31978058,47.06235327)(6.25630402,47.60922827)
\curveto(6.19771027,48.15610327)(6.08052277,48.61997046)(5.90474152,49.00082984)
\curveto(5.73384308,49.3768064)(5.49946808,49.66000952)(5.20161652,49.85043921)
\curveto(4.90864777,50.0408689)(4.5302298,50.13608374)(4.06636261,50.13608374)
\curveto(3.60737824,50.13608374)(3.22651886,50.0408689)(2.92378449,49.85043921)
\curveto(2.62593293,49.66000952)(2.38911652,49.37192359)(2.21333527,48.9861814)
\curveto(2.04731964,48.62485327)(1.93257355,48.15366187)(1.86909699,47.57260718)
\curveto(1.81050324,46.99155249)(1.78120636,46.35434546)(1.78120636,45.66098609)
\curveto(1.78120636,44.89926734)(1.80806183,44.2620603)(1.86177277,43.74936499)
\curveto(1.91548371,43.23666968)(2.0302298,42.7776853)(2.20601105,42.37241187)
\curveto(2.36714386,41.99155249)(2.59419464,41.70102515)(2.88716339,41.50082984)
\curveto(3.18501496,41.30063452)(3.57808136,41.20053687)(4.06636261,41.20053687)
\curveto(4.52534699,41.20053687)(4.90620636,41.29575171)(5.20894074,41.4861814)
\curveto(5.51167511,41.67661109)(5.74605011,41.96469702)(5.91206574,42.35043921)
\closepath
}
}
{
\newrgbcolor{curcolor}{0 0 0}
\pscustom[linestyle=none,fillstyle=solid,fillcolor=curcolor]
{
\newpath
\moveto(61.29719757,40.21909155)
\lineto(55.39387726,40.21909155)
\lineto(55.39387726,41.3323728)
\lineto(57.66438507,41.3323728)
\lineto(57.66438507,48.64194312)
\lineto(55.39387726,48.64194312)
\lineto(55.39387726,49.63803687)
\curveto(55.70149445,49.63803687)(56.03108429,49.66245093)(56.38264679,49.71127905)
\curveto(56.73420929,49.76498999)(57.00032257,49.84067359)(57.18098664,49.93832984)
\curveto(57.40559601,50.06040015)(57.58137726,50.21420874)(57.70833039,50.39975562)
\curveto(57.84016632,50.5901853)(57.91584992,50.84409155)(57.93538117,51.16147437)
\lineto(59.07063507,51.16147437)
\lineto(59.07063507,41.3323728)
\lineto(61.29719757,41.3323728)
\lineto(61.29719757,40.21909155)
\closepath
}
}
{
\newrgbcolor{curcolor}{1 0 0}
\pscustom[linestyle=none,fillstyle=solid,fillcolor=curcolor]
{
\newpath
\moveto(4.64285707,56.38153662)
\curveto(4.64285707,55.09944704)(3.60351811,54.06010809)(2.32142853,54.06010809)
\curveto(1.03933895,54.06010809)(-0.00000001,55.09944704)(-0.00000001,56.38153662)
\curveto(-0.00000001,57.6636262)(1.03933895,58.70296516)(2.32142853,58.70296516)
\curveto(3.60351811,58.70296516)(4.64285707,57.6636262)(4.64285707,56.38153662)
\closepath
}
}
{
\newrgbcolor{curcolor}{0 0 0.50196081}
\pscustom[linestyle=none,fillstyle=solid,fillcolor=curcolor]
{
\newpath
\moveto(57.69944429,56.38153662)
\curveto(57.69944429,55.09944704)(56.66010533,54.06010809)(55.37801575,54.06010809)
\curveto(54.09592617,54.06010809)(53.05658721,55.09944704)(53.05658721,56.38153662)
\curveto(53.05658721,57.6636262)(54.09592617,58.70296516)(55.37801575,58.70296516)
\curveto(56.66010533,58.70296516)(57.69944429,57.6636262)(57.69944429,56.38153662)
\closepath
}
}
\end{pspicture}

		\end{figure}

		\pause
		$$\partial [01] = [0] - [1]; \partial [012] = [01] - [02] + [12];$$
		$$\partial [x_1 x_2\ldots x_n] = \sum_{i=1..n}(-1)^{n-i}[x_1\ldots{\color{red}x_i}\ldots x_n]$$

		\column{0.45\textwidth}
		\pause
		Вопросы:
		\begin{itemize}
		\item Чему равно $\partial [0123]$?
		\item Проверить $\partial\partial [012]=0$
		\item Почему $\partial\partial=0$?
		\end{itemize}
		\pause
		\begin{block}{Определение}
		\begin{itemize}
		\item Симплициальное пространство состоит из симплексов
		\item Вместе с каждым симплексом включает все его грани
		\item Два симплекса не могут иметь более 1 общей грани!
		\item Симплициальный комплекс состоит из свободных (абелевых) групп
		\item По одной образующей $n$-ной компоненты на каждый $n$-симплекс
		\item Дифференциал определяется значениями на порождающих свободной группы
		\end{itemize}
		\end{block}
		\end{columns}
	}

	\frame {
		\frametitle{Симплициальные гомологии. Пример}
		\large
		\begin{columns}
		\column{0.6\textwidth}
		Пример --- сфера.
		\onslide<2-> Тут \begin{itemize}
		\item 0 симплексов $\Delta_3$
		\item 4 симплекса $\Delta_2$
		\item 6 симплексов $\Delta_1$
		\item 4 симплекса $\Delta_0$
		\end{itemize}

	\kern2cm
		\begin{tabular}{c c c c c c c c c c  }
		&&& 0 && 1 && 2 \\
		$K$ & $0$ & $\leftarrow$ & ${\mathbb Z}^4$ & $\leftarrow$ & ${\mathbb Z}^6$ & $\leftarrow$ & ${\mathbb Z}^4$ & $\leftarrow$ & $0$ \\
		\onslide<2-> $Z$&&& \onslide<3-> $K_0$ && \onslide<5-> $B_1$ &&  \onslide<4-> $\mathbb Z$ \\ 
		\onslide<2-> $B$&&& \onslide<5-> ${\mathbb Z}^3$ &&  \onslide<4-> $B_1$ && \onslide<3-> 0 \\ 
		\onslide<2-> $H$&&& \onslide<6-> $\mathbb Z$ && \onslide<6-> $0$  && \onslide<6-> $\mathbb Z$ \\ 
		\end{tabular}

		\column{0.3\textwidth}
		\onslide<1->
		\begin{figure}
		    \centering
		    \def\svgwidth{\columnwidth}
		    \input{simplex.pdf_tex}
		\end{figure}
		\onslide<7->
		Сфера и проективная плоскость. С ней сложнее
		\begin{figure}
		    \centering
		    \def\svgwidth{\columnwidth}
		    \input{sipl-sphere.pdf_tex}
		\end{figure}
		\end{columns}
	}

	\frame {
		\frametitle{Карта}
		
		\begin{figure}
		    \centering
		    \def\svgwidth{1\columnwidth}
		    \input{allhom1.pdf_tex}
		\end{figure}
	}

\frame {
\large
		\frametitle{Сингулярные гомологии}
		
		\begin{itemize}
		\item У произвольного топологического пространства
		\item Не требуют никакой особой структуры
		\item Оперируют с бесконечнопорождёнными (несчётнопорождёнными) группами, вычислять гомологии очень сложно
		\item Просто доказать условие комплекса $\partial\partial=0$
		\item Инвариантность очевидна
		\item Можно доказать инвариантность для других видов гомологий, путём сведения к сингулярным
		\end{itemize}

		\pause

		Построение

		\begin{itemize}
		\item В каждой размерности $n\geqslant0$ берётся один симплекс $\Delta_n$
		\item Рассматриваются всевозможные непрерывные отображения $n$-симплекса в топологическое пространство $X$. Каждое отображение~--- образующая $K_n$. \pause
		\item Оператор $\partial$ от отображения $f:\Delta_n\to X$ берёт сумму ограничений $f$ на всевозможные грани $\Delta_n$ с соответствующими знаками.
		\item Далее, комплекс по образующим и заданию $\partial$ на образующих строится точно так же, как и симплициальный
		\end{itemize}
	}

	\frame {
\large
		\frametitle{Сингулярные гомологии. Пример}
		Вычислим сингулярные гомологии \ldots \onslide<2-> точки ($*$). 

	\kern1cm
		\begin{tabular}{c c c c c c c c c c  c c c c c c }
		&&& 0 && 1 && 2 && 3 && 4 & \ldots \\
		$K$ & \onslide<3-> $0$ & $\leftarrow$ & $\mathbb Z$ & $\leftarrow$ & $\mathbb Z$ & $\leftarrow$ & $\mathbb Z$ & $\leftarrow$ & $\mathbb Z$ & $\leftarrow$ & $\mathbb Z$ & $\leftarrow$ \\
		\onslide<2-> $Z$&&& \onslide<3-> $\mathbb Z$ \onslide<7->  && $\mathbb Z$  && 0 && $\mathbb Z$ && 0 \\ 
		\onslide<2-> $B$&&& \onslide<7-> 0 && $\mathbb Z$ && 0 && $\mathbb Z$ && 0 \\ 
		\onslide<2-> $H$&&& \onslide<8-> $\mathbb Z$ && 0 && 0 && 0 && 0 \\ 
		\end{tabular}

	\kern1cm
		\begin{itemize}
		\item \onslide<3-> Для каждого $n$ есть только одно отображение $f_n: \Delta_n\to*$
		\item \onslide<4-> $\partial f_n = $ \onslide<5-> $ f_{n-1} - f_{n-1} + f_{n-1} - \ldots + (-1)^{n} f_{n-1}$
		\item \onslide<6-> Таким образом $\partial f_n = 0$, если $n$~--- нечётное, и $f_{n-1}$, если чётное
		\item \onslide<7-> $\partial f_n = 0$ означает, что $f_n\in Z_n$ и $B_{n-1} = 0$
		\end{itemize}
	}

	\frame {
		\frametitle{Карта}
		
		\begin{figure}
		    \centering
		    \def\svgwidth{1\columnwidth}
		    \input{allhom2.pdf_tex}
		\end{figure}
	}


	\frame {
		\large
		\frametitle{Ликбез по геометрии}
		Сфера $\mathbb S^n$~--- граница $n+1$-мерного открытого шара (в метрическом смысле). Открытый шар $O^n$ топологически $n$-мерное евклидого пространство. По определению, $O^0 = *$.

		\pause Вопросы:
		\begin{itemize}
		\item Что такое 1-мерная сфера? \pause Ответ: окружность
		\item Что такое 0-мерная сфера? \pause Ответ: две точки
		\item Что такое -1-мерная сфера? \pause Ответ: пустое множество
		\end{itemize}

		Топологическое пространство можно профакторизовать по замкнутому множеству, т. е. склеить это замкнутое множество в одну точку.

		\begin{itemize}
		\item Если профакторизовать замкнутый шар $D^n$ по границе ($n>0$), то получим \pause сферу $\mathbb S^n$\pause
		\item Если открытый шар  $O^n$ ~--- открытое подмножество пространства $X$, то факторизация $X$ по дополнению до шара~--- это $\mathbb S^n$. Причём это верно даже для $n=0$.
		\end{itemize}

		Можно определить индекс отображения $\mathbb S^n\to\mathbb S^n$. Неформально~--- сколько раз одна сфера "наматывается"\ на другую. Индекс может быть отрицательным.
	}

	\frame {
		\frametitle{Клеточные гомологии}

		\begin{columns}
		\column{0.40\textwidth}

		\begin{itemize}
		\item У топологического пространства с заданным разбиением на клетки
		\item Требуют задания явной структуры
		\item Гомологии вычислять легко и приятно (нет)
		\item Инвариантность и $\partial\partial$ доказать сложно, но можно свести к сингулярным, откуда это следует
		\end{itemize}
		
		\large
		\onslide<5->
		\begin{block}{Пример}
		Cфера $\mathbb S^2$. Можно собрать её из \onslide<6-> одной 2-мерной клетки ($O_2$) и одной 0-мерной (точки). Весь круг --- граница $O_2$ --- отображается в точку.
		\end{block}

		\column{0.55\textwidth}
		\large
		\onslide<2->
		\begin{block}{Определение CW-пространства}
		\begin{itemize}
		\item Клетка размерности $n$~--- непрерывное \textbf{вложение} открытого шара $O^n$ в наше пространство $X$.
		\item Клетки не пересекаются. $X$ это объединение всех клеток всех размерностей.
		\item $n$-остов объединение всех клеток размерности $n$ и меньше, обозначение $X_n$
		\onslide<3->
		\item Вложение клетки продолжается по непрерывности на её границу ($n-1$-мерную сферу). Образ границы обязан лежать в $n-1$-остове и пересекаться лишь с конечным числом клеток размерности $n-1$ и меньше. 
		\onslide<4->
		\item При построении CW-комплекса образующими $K_n$ будут $n$-мерные клетки
		\end{itemize}
		\end{block}

		\end{columns}
	}

	\frame {
		\frametitle{Клеточные гомологии. Пример пространства}
		\Large

		\begin{block}{Пример}
		Проективная плоскость $\mathbb RP^2$. Сфера, у которой диаметрально противоположные точки отождествлены. 

		\onslide<2-> Если удалить все "дубликаты" \ точек, получим \onslide<5->открытый круг, интервал и точку.
		Т. е. по одной клетке размерностей 0, 1, 2.
		\end{block}
		
		\begin{figure}
		    \centering
		    \onslide<1->\def\svgwidth{.2\columnwidth}
		    \input{rp2-1a.pdf_tex}
		    \onslide<2->\def\svgwidth{.2\columnwidth}
		    \input{rp2-2a.pdf_tex}
		    \onslide<3->\def\svgwidth{.2\columnwidth}
		    \input{rp2-3a.pdf_tex}
		    \onslide<4->\def\svgwidth{.2\columnwidth}
		    \input{rp2-4a.pdf_tex}
		\end{figure}
	}

	\frame {
		\frametitle{Клеточные гомологии}
		\Large

		
		\begin{block}{Граничный гомоморфизм}
		\begin{itemize}
		\item Чтобы построить $\partial$ нужно определить, сколько раз каждая $n-1$-клетка $y: O^{n-1}\to X$ входит в границу $n$-клетки $x: O^n\to X$.
		\pause\item Для этого вспомним, что отображение $x$ продолжается на границу шара $O^n$, которая является сферой $\mathbb S^{n-1}$. Получаем отображение $\bar x: \mathbb S^{n-1}\to X_{n-1}$ в $n-1$-мерный остов.
		\pause\item Если клетка $y$ не пересекается с $\Ima\bar x$, то она не входит в $\partial x$.
		\pause\item Иначе, профакторизуем $n-1$-остов по дополнению до $y$. Как утверждалось, получим снова сферу $\mathbb S^{n-1}$ и отображение из $\hat y:X_{n-1}\to\mathbb S^{n-1}$.
		\pause\item Беря композицию отображения границы $x$ в остов и отображения факторизации получим отображение $\hat y\circ\bar x:\mathbb S^{n-1}\to\mathbb S^{n-1}$
		\pause\item Его степень - индекс вхождения клетки $y$ в $\partial x$.
		\pause\item Как и раньше, $\partial$ в общем виде задаётся значениями на образующих
		\end{itemize}
		\end{block}
	}


	\frame {
		\frametitle{Клеточные гомологии. Пример}
		\Large
		\begin{block}{Пример}
		Сфера $\mathbb S^2$. \pause Как было сказано, задаётся двумя клетками: двумерной и 0-мерной.
		\end{block}

		\pause
		\begin{tabular}{c c c c c c c c c c  }
		&&& 0 && 1 && 2 \\
		$K$ & $0$ & $\leftarrow$ & ${\mathbb Z}$ & $\leftarrow$ & 0 & $\leftarrow$ & ${\mathbb Z}$ & $\leftarrow$ & $0$ \\
		\pause
		$Z$&&&  ${\mathbb Z}$ && 0 &&  $\mathbb Z$ \\ 
		$B$&&&  $0$ &&  $0$ &&  0 \\ 
		\pause
		$H$&&& $\mathbb Z$ &&  $0$  &&  $\mathbb Z$ \\ 
		\end{tabular}
		
		\begin{figure}
		    \centering
		    \onslide<1->\def\svgwidth{.2\columnwidth}
		    \input{rp2-1a.pdf_tex}
		\end{figure}
	}

	\frame {
		\frametitle{Клеточные гомологии. Пример 2}
		\large
		\begin{block}{Пример}
		Проективная плоскость $\mathbb RP^2$. \pause Как было сказано, задаётся тремя клетками: двумерной, 1-мерной и 0-мерной.
		\pause Тут придётся подумать, как устроены граничные операторы.

		\pause Граница круга два раза наматывается на одномерную клетку. Поэтому коэффициент будет равен двум.

		\pause Граница полуокружности дважды переодит в 0-мерную клетку, но с разными знаками, поэтому коэффициент 0. Это очевидно также из включения $B_1\subset Z_1$
		\end{block}

		\Large
		\begin{tabular}{c c c c c c c c c c  }
		\onslide<2->&&& 0 && 1 && 2 \\
		$K$ & $0$ & $\leftarrow$ & ${\mathbb Z}$ & $\leftarrow$ & ${\mathbb Z}$ & $\leftarrow$ & ${\mathbb Z}$ & $\leftarrow$ & $0$ \\
		\onslide<2->$Z$&&& \onslide<2-> ${\mathbb Z}$ && \onslide<5->${\mathbb Z}$ &&  \onslide<4-> 0 \\ 
		\onslide<2->$B$&&& \onslide<5-> $0$ &&  \onslide<4-> $2{\mathbb Z}$ &&  \onslide<2-> 0 \\ 
		\onslide<2->$H$&&& \onslide<6-> $\mathbb Z$ &&  $\mathbb Z_2$  &&  $0$ \\ 
		\end{tabular}

		\large
		\onslide<7-> Верхняя группа гомологий 0, это признак того, что поверхность неориентируема.
		
		\begin{figure}
		    \centering
		    \onslide<1->\def\svgwidth{.1\columnwidth}
		    \input{rp2-1a.pdf_tex}
		    \onslide<1->\def\svgwidth{.1\columnwidth}
		    \input{rp2-2a.pdf_tex}
		    \onslide<1->\def\svgwidth{.1\columnwidth}
		    \input{rp2-3a.pdf_tex}
		    \onslide<1->\def\svgwidth{.1\columnwidth}
		    \input{rp2-4a.pdf_tex}
		\end{figure}
	}

	\frame {
		\frametitle{Карта}
		
		\begin{figure}
		    \centering
		    \def\svgwidth{1\columnwidth}
		    \input{allhom3.pdf_tex}
		\end{figure}
	}

	\frame {
		\frametitle{Что не затронули}
		\Large
		
		Что ещё интересного в основах теории гомологий
		\begin{itemize}
		    \pause\item Когомологии
		    \pause\item Относительные гомологии
		    \pause\item Другие теории гомологий		    
		    \pause\item Гомотопическая инвариантность
		    \pause\item Функториальность
		    \pause\item Гомологическую точную последовательность
		    \pause\item Умножение в когомологиях
		    \pause\item Ещё много всего
		\end{itemize}
	}
\end{document}